\section{Introduction}\label{sec:introduction}

Using indexed data types and dependent types it is possible to encode
logic in algebraic data types to enforce invariants.
Encoding invariants in data types allows us to write correct-by-construction
software.
However, it makes code reuse very hard.
Functions which work on one data type do not work on a similar data
type with added invariants.

Algebraic ornaments capture the relationship between two data types
with the same recursive structure.
Ornaments can be used to transform a functions which work on the first
data type to functions which work on the second.
For instance to transform addition on natural numbers to append on lists.

This topic has been explored and formalised in Agda.
These implementations require you to describe your data types within a
universe of descriptions.
The descriptions can be interpreted as a data type, but this approach
is not ideal for programming.

In this project we will work towards a more practical way to use
ornaments in Agda.
We want to be able to define data types as usual and define ornaments
on these data types.
Ornamenting a data type should result in a newly defined data type.
Additionally, it should be possible to transform functions using these
ornaments.

