\section{Introduction}\label{sec:introduction}

With the introduction of indexed data types (GADTs), functional
programmers are now able to capture invariants of their programs with
their data.
Encoding invariants in data types is yet another step towards writing
correct-by-construction software.
However, it can be a significant obstacle for code reuse.
Functions which work on one data type do not work on a similar data
type with added invariants.
For example, vectors are the same as lists with an extra length index,
but the append (++) function for lists does not work for vectors.

Algebraic ornaments capture the relationship between two data types
with the same recursive structure.
Ornaments can be used to guide the transformation of functions which
work on the first data type to functions which work on the
second\cite{dagand14-transporting, williams14}.
For instance to transform addition on natural numbers to append on lists.

This topic has been explored and formalised using category theory
\cite{dagand12, kogibbons13} and in Agda \cite{dagand14-transporting,
  dagand14-essence}.
The Agda implementations require you to describe your data types
within a universe of descriptions.
These descriptions can be interpreted as a data type, but this approach
is not ideal for programming.

In this project we will work towards a more practical way to use
ornaments in Agda.
We want to be able to define data types as usual and define ornaments
on these data types.
Ornamenting a data type should result in a newly defined data type.
